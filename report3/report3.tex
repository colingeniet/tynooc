\documentclass{article}

\usepackage[utf8]{inputenc}
\usepackage[T1]{fontenc}

\usepackage{enumitem}

\begin{document}
\title{Tynooc --- A Tycoon clone \\ \large{Project report --- part 3}}
\author{Valentin Maestracci \and Yoan Geran \and Colin Geniet}
\maketitle
\tableofcontents

\section{Save Files System}
We chose to implement the saves system with Scala built-in binary serialization
(trait \verb|Serializable|). This makes saving relatively easy, as most of it
is done automatically by the JVM : the whole state of the game logic engine is
saved in one file.

The main problem however is that ScalaFX components can not be serialized.
Thus, only the game engine, and not the GUI must be saved.
Thankfully, the separation was clear enough in our game to avoid problems.
The only difficulties were in saving ScalaFX properties and bindings
(which are not serializable). This was achieved by only saving the property
values, as only the GUI listen to properties, making unnecessary to save any
listener associated with a property.

\section{AI}
\subsection{Genetic AI}
The most interesting AI we created is a so-called genetic AI.
The basic idea is to generate `interesting' travels by generating a number
of random paths, then select the bests based on some heuristic.
To improve results, this is done in two phases:
\begin{itemize}
  \item Several random paths of fixed length are generated.
  \item A few are selected based on the heuristic.
  \item The selected paths are randomly `mutated'.
  \item The best of the mutated paths is selected.
\end{itemize}

The heuristic we use estimates the cost and rewards of the trip based on current
world status (goods availability and prices, \dots).

\subsection{Scripted AI}
Another AI we implemented relies on a more fixed behavior:
when starting a new travel, the town with the most goods to export is selected
as the depart. Then, the town with the most goods requested among the ones exported
is selected as the destination. Thus, the behavior is deterministic, unlike the
genetic AI.


\section{Economy rework}
\subsection{New economy}
The economy system was reworked since part 2 to make it simpler and more efficient.
Previously, at each economic tick (separated by a few in game hours), for every
pair of towns and every goods, a quantity of that goods to be transported
between those two towns was calculated.

Now, every town simply calculates an amount to export and import for each goods.
The choice of transporting goods from a town to another is delayed until a
vehicle actually travels. Only then is the amount to transport calculated.

\subsection{Missions}
An additional feature is the addition of missions: under some circumstances
(important requests from a town), missions are generated, requiring the player
to transport a given amount of goods between two given towns. Upon completion,
the player will earn an additional reward compared to a normal trip.

Missions have a deadline that negates the reward if not respected. The player
may choose to accept or decline any mission they are proposed.


\end{document}

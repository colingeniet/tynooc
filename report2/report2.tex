\documentclass{article}

\usepackage[utf8]{inputenc}
\usepackage[T1]{fontenc}

\usepackage{enumitem}

\begin{document}
\title{Tynooc --- A Tycoon clone \\ \large{Project report --- part 2}}
\author{Valentin Maestracci \and Yoan Geran \and Colin Geniet}
\maketitle

\tableofcontents

\section{Modifications to part 1}
Some changes were made to the first part of the project to allow better introduction of new features for part 2.

\subsection{Use of Properties}
The first modification we made was to rewrite the GUI using scalaFX property system,
replacing the use of \verb|draw| functions.
Thus, the tick by tick update system is now limited to the game logic, with the GUI automatically updating.

The main goal of this change was to simplify the GUI code, and remove some bugs due to the GUI
not correctly updating. While it took some time to apply this change, it made developing the interface
for the second part much easier.

\subsection{Optimizations}
Other major changes were all related to optimization, and were required in order to be able to use
bigger maps (such as the U.K. one). Those changes include
\begin{itemize}
\item Optimization of stupid code: removal of redundant calculations, active waiting\dots
\item Reduced rate of economics calculations: the calculations used to be made at every game tick (i.e. at 60 Hz),
which is way more than needed, and way too much on a big map.
Now, only vehicle positions are updated at that rate, and the generation of passengers is done at a much slower rate.
\end{itemize}


\section{Introduction of new features}
\subsection{New vehicles}
The introduction of new vehicle types did not pose any major challenge.
It was mostly a matter of generalizing the patterns already introduced for trains
and factorizing code.

\subsection{Goods and factories} 

\end{document}
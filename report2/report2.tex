\documentclass{article}

\usepackage[utf8]{inputenc}
\usepackage[T1]{fontenc}

\usepackage{enumitem}

\begin{document}
\title{Tynooc --- A Tycoon clone \\ \large{Project report --- part 2}}
\author{Valentin Maestracci \and Yoan Geran \and Colin Geniet}
\maketitle

\tableofcontents

\section{Modifications to part 1}
Some changes were made to the first part of the project to allow better introduction of new features for part 2.

\subsection{Use of Properties}
The first modification we made was to rewrite the GUI using scalaFX property system,
replacing the use of \verb|draw| functions.
Thus, the tick by tick update system is now limited to the game logic, with the GUI automatically updating.

The main goal of this change was to simplify the GUI code, and remove some bugs due to the GUI
not correctly updating. While it took some time to apply this change, it made developing the interface
for the second part much easier.

\subsection{Generalization and factorization}
Secondly, we spent a significant time generalizing classes and patterns from part 1 to accommodate for part 2.
For instance
\begin{itemize}
\item All the code regarding trains was separated in code general to vehicles, and code specific to trains,
in order to introduce new vehicles.
\item The model system that defines various engine and carriage models was highly generalized and later
reused for other vehicles, but also factories, stations \dots
\end{itemize}
Those generalizations allowed significant factorization in the code for second part.
For example, some of the code to display vehicles statistics is shared with the code to display facility statistics,
despite the obvious differences.

\subsection{Optimizations}
Other major changes were all related to optimization, and were required in order to be able to use
bigger maps (such as the U.K. one). Those changes include
\begin{itemize}
\item Optimization of stupid code: removal of redundant calculations, removal of active waiting for passengers\dots
\item Reduced rate of economics calculations: the calculations used to be made at every game tick (i.e. at 60 Hz),
which is way more than needed, and way too much on a big map.
Now, only vehicle positions are updated at that rate, and the generation of passengers is done at a much slower rate.
\end{itemize}


\section{Introduction of new features}
\subsection{New vehicles}
After the generalization of the first part, implementing new vehicle types did not pose any problem.
Similarly, adding connection types (rails, road, \dots) and the matching restrictions on vehicle moves was quite easy.

\subsection{Stations}
Adding stations and the restriction that vehicles need an appropriate station to stop at a city did not pose problems. 

We were however unable to implement a meaningful game mechanic to restrict station usage (i.e.\ a limit on the number of vehicles).
Indeed, vehicles in our game only stop for one game tick upon reaching a city, meaning that there
will statically never be more than one vehicle stopped at a city at any given time, making such restrictions meaningless.

While we may have considered more complicated game mechanics that would take time into account,
we did not have time to implement any such thing.

\subsection{Goods and prices}
One of the features of the goods system we implemented is a trait system that allows to define some characteristics
for goods. In particular, this system is used a general framework to implement mechanics such as food rotting or liquids evaporating.

On the world scale, goods are stored by each town, consumed by the population, and produced (but also used) by factories.
Goods have different prices in each town, which change based on the relative stocks of the towns, and on the consumption.
This way, prices will significantly in a town if stocks are short, and consumption is high.

Towns will decide to export goods to other towns based on price differences (exportation is towards towns with higher prices).
Any company (i.e.\ the player or the AIs) will then be able to transport the goods to export, earning the prices difference.

\subsection{Factories}
As for factories, the most interesting feature is the production optimization.
Our factories have a set of productions.
Each production consumes a fixed quantity of some goods, and then after a delay produces a fixed quantity of other goods.

At any time, each factory can run one of its productions.
The choice of production is done to optimize the gain from production: the expected gain is calculated based on current prices,
and the production with maximum gain is chosen.

This mechanic is relatively simple, and allowed us having to implement optimization without a linear optimization algorithm,
which we lacked time to do.

\subsection{Trips}
Our previous trip system only allowed to order a vehicle to travel to a given town.
To implement a complex trips system, we chose to keep the old system as the underlying layer.
Complex travels are then composed of a series of simple `go to' instructions, which may additionally be repeated.
In addition, a vehicle can also be instructed to wait during that sequence.

This separation in two layer had the major advantage of not requiring any change to the existing travel system,
and thus to the part of the logic engine depending on it.

\end{document}
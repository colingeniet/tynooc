\documentclass{article}

\usepackage[utf8]{inputenc}
\usepackage[T1]{fontenc}

\usepackage{enumitem}

\begin{document}
\title{Tynooc --- A Tycoon clone \\ \large{Project report --- part 1}}
\author{Valentin Maestracci \and Yoan Geran \and Colin Geniet}
\maketitle

\tableofcontents

\section{General program organization}
The game is split into two main parts : the game logic engine (or engine), and the interface (or gui):
\begin{itemize}
\item The game engine simulates the world, that is simulates passengers and trains movements.
\item The gui interacts with the engine by transmitting player actions to the engine, 
and printing back the world content (trains, population\dots).
\end{itemize}

This pattern is similar to a simplified Model-View-Controller design pattern:
\begin{itemize}[noitemsep]
\item The model corresponds to the game engine.
\item The view corresponds to the gui.
\item The controller is implemented by the graphic library (scalaFX), and is therefore mostly hidden in the game code.
The controller setup, in the form of of scalaFX event handlers, is integrated in the gui.
\end{itemize}



\section{Game logic engine}
The game engine defines all game objects (towns, routes, trains, carriages ...), and simulates world evolution.
World simulation uses a step by step logic : at each step, trains advance based on the time passed since last step,
and towns generate passengers.

\subsection{Passengers}
Passenger generation raised several questions regarding the way town population and passengers are represented.

Our initial idea, aimed at relatively small population, was to simulate each person individually.
Each person would then have a position, and decide to go on trips based on various parameters.
Obviously, this approach poses efficiency problems as the world population grows bigger.

An alternative that we considered is to simulate groups of persons instead, in a similar way.
What was the simulation of one person would then become the simulation of a group of a few hundred of persons,
considered to have similar behavior.

However we preferred to redo this system. Only the population size and the number of passengers are stored
for each town. This obviously allows the simulation to be significantly more efficient than with the previous method.

In an effort to keep some of the extra features of the first version, we did however decide to add
a notion of passenger `status'. Passengers of different status have slightly different simulation rules,
in their train choices for instance. The way this is implemented is by keeping population records for each 
status separately.

\section{GUI}
The main problem we faced in implementing the game gui with scalaFX was to display constantly changing objects ---
typically trains on the world map. A typical way to handle is to use a main loop which continuously updates the interface.
However, the scalaFX library is oriented towards the an event handling logic
in which changes in the interface come as a result of an event (user action, ...).

We decided to solve this problem by combining the two methods, 
and using the most appropriate one, depending on the situation :
\begin{itemize}
\item Interface elements that only change as a result of user actions (typically menus) 
are implemented through event handling.

\item Changing elements (trains, towns population) are continuously refreshed by a loop.
To integrate this in scalaFX event handling logic, this loop is ran in a background thread,
and merely request scalaFX to update the interface.

Of course, to avoid overloading scalaFX with update requests, care was taken to keep updates as light as possible.
In particular, object creation are kept as minimal as possible, with most updates being limited to text fields
or screen coordinates changes.
\end{itemize}

An alternative system we also considered is to make use of scalaFX properties.
The idea is that any update to a property value is an event, which can result in interface updates.
Using properties in the game engine would then allow any change in the world to result in the corresponding
gui parts being updated.


\section{Additional features}
We started developing the following features. Although they are still very much in a development state,
we deemed them interesting enough to add them in this first part of the project.

\subsection{Map files}
We created a simple text file format to represent world data (i.e.\ towns and routes), 
in order to avoid having to hard code the world map.
Currently, the game simply loads the file \verb|map/Map|.
This will obviously be improved in future versions, to allow loading other map files.
We might also add a map editor.

Additionally, the current regex-based parser will also be rewritten,
mostly to allow better error handling.

\subsection{AI}
In order to demonstrate the possibility to have several `players' in the same game,
we developed a very simple game AI. For now, it simply creates trains and randomly send them on trips.


\end{document}
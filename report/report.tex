\documentclass{article}

\usepackage[utf8]{inputenc}
\usepackage[T1]{fontenc}

\usepackage{enumitem}

\begin{document}
\title{Tynooc --- A Tycoon clone \\ \large{Project report --- part 1}}
\author{Valentin Maestracci \and Yoan Geran \and Colin Geniet}
\maketitle

\tableofcontents

\section{General program organization}
The game is split into two main parts : the game logic engine (or engine), and the interface (or gui) :
\begin{itemize}
\item The game engine simulates the world, that is simulates passengers and trains.
\item The gui interacts with the engine by transmitting player actions to the engine, 
and printing back the world content (trains, population\dots).
\end{itemize}

This pattern can be compared to a simplified Model-View-Controller pattern :
\begin{itemize}[noitemsep]
\item The model corresponds to the game engine.
\item The view corresponds to the gui.
\item The controller is implemented by the graphic library (scalaFX), and is therefore mostly hidden in the game code.
The controller setup, in the form of of scalaFX event handlers, is integrated in the gui.
\end{itemize}



\section{Game logic engine}
The game engine (package \verb|logic|) defines all game objects (towns, routes, trains, ...), and simulates world evolution.
World simulation uses a step by step logic : at each step, 


\section{GUI}
The main problem we faced in implementing the game gui with scalaFX was to display constantly changing objects ---
typically trains on the world map. A typical way to handle would be a through a main loop which continuously updates
the interface.

This, however, conflicts with the scalaFX event handling logic in which changes in the interface come as a result of an event
(user action, ...).

\end{document}